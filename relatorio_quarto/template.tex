\usepackage{xcolor}
\usepackage{fontspec}
\usepackage{sectsty}
\usepackage{geometry}
\usepackage{pagecolor}
\usepackage{background}
\usepackage{tocloft}
\usepackage{fancyhdr}
\usepackage{titlesec}
\usepackage{tikz}
\usepackage{graphicx}
\usepackage{indentfirst}
\usepackage{etoolbox}
\usepackage{pdfpages}
\usepackage{enumitem}

\setlength{\footskip}{30pt}
%\usepackage[style=apa]{biblatex}
%\addbibresource{referencias.bib}


% Custom bullet styles for nested lists
\setlist[itemize,1]{label=$\bullet$}   % Level 1: •
\setlist[itemize,2]{label=$\circ$}     % Level 2: ◦
\setlist[itemize,3]{label=$\diamond$} % Level 3: ▶

% Adicionar quebra de página antes de cada seção
\AddToHook{cmd/section/before}{\clearpage}

% definir fontes segundo padroes 
%\setmainfont{Trebuchet MS}
\setsansfont{Times New Roman}
\newfontface\tbBold{Times New Roman-Bold}

% estilo da capa do relatório
\definecolor{corTextoCapa}{HTML}{FFFFFF}
\definecolor{corTitulo}{HTML}{46291c}
\definecolor{corSubsDois}{HTML}{000000} % Preto

\pagenumbering{Roman}

% Configurar cabeçalhos e rodapés
\pagestyle{fancy}
\fancyhf{}
\fancyfoot[R]{\thepage} % Fonte e cor da numeração das páginas

\fancypagestyle{plain}{
  \fancyhf{}
  \renewcommand{\headrulewidth}{0pt}
  \fancyfoot[R]{\thepage} % Garantir alinhamento à direita
  \fancyfootoffset{0pt}
}

% No preâmbulo (após carregar fancyhdr):
\fancypagestyle{tocfirstpage}{
  \fancyhf{}
  \renewcommand{\headrulewidth}{0pt}
  \fancyfoot{} % Remove TODOS os rodapés
}

% Forçar estilo personalizado APENAS na primeira página do TOC
\AfterEndPreamble{
  \addtocontents{toc}{\protect\thispagestyle{tocfirstpage}}
}

% Estilos globais do doc
\allsectionsfont{\color{corSubsDois}}
\sectionfont{\textbf{\color{corSubsDois}}}
%\sectionfont{\color{corSubsDois}}
%\subsectionfont{\color{corSubsDois}}
%\subsubsectionfont{\color{corSubsDois}}
%\paragraphfont{\color{corSubsDois}\bfseries}
%\subparagraphfont{\color{corSubsDois}\bfseries}

\setlength{\parindent}{1.5em} % Recuo de parágrafo
\setlength{\parskip}{1.5pt} % Sem espaço adicional entre parágrafos
\linespread{1.5} % Espaçamento entre linhas

% Margens
\geometry{a4paper, top=2.5cm, bottom=2.5cm, left=2.5cm, right=2.5cm}

% add a imagem de background na página da capa (template vazio)
\backgroundsetup{
  scale=1,
  angle=0,
  position=current page.center,
  vshift=0pt,
  hshift=0pt,
  contents={\includegraphics[width=0.8\paperwidth, height=1.0\paperheight]{modelostex/CAPA_PROVISORIA.pdf}}
}


% Estilos específicos da CAPA (EDITAR titulo, autor e data da CAPA AQUI)
\renewcommand{\familydefault}{\sfdefault}
\renewcommand{\maketitle}{
  \begin{titlepage}
    \newgeometry{top=5cm, bottom=5cm, left=4cm, right=4cm}
    %\pagecolor{corFundoCapa}
    \color{corTextoCapa}
    \centering

    %Espaço vertical ajustado para o título
    \vspace*{4em}
    \Huge{\textbf{\textsf{\textcolor{corTitulo}{}}}} % Ajuste para usar \@title para o título

    % Espaço vertical para o autor
    \vfill
    \Large{\textbf{\textsf{\textcolor{corTitulo}{\@}}}} % Ajuste para usar \@author para o autor
    \vspace{3em}

    % DATA
    \Large{\textbf{\textsf{\textcolor{corTitulo}{\@}}}}

    \restoregeometry
    \pagecolor{white}\color{black}
  \end{titlepage}


% Definição da cor do título do TOC P/ QUE ACOMPANHE VERDE
\definecolor{tocTitleColor}{HTML}{000000}  % Defina a cor desejada

% Estilo do título do Toc
\renewcommand{\cfttoctitlefont}{\color{tocTitleColor}\Huge\bfseries}

% Ajustar a distância entre o título do Toc e 1å seção
\setlength{\cftaftertoctitleskip}{20pt}

% retirar as bolinhas entre o texto e o número da página no toc
%\renewcommand{\cftdot}{}

% Alinhar o TOC à esquerda e ajustar espaçamento
%\renewcommand{\cftsecindent}{0pt} % Remove a indentação das seções
%\renewcommand{\cftsubsecindent}{0pt} % Remove a indentação das subseções
%\renewcommand{\cftsubsubsecindent}{0pt} % Remove a indentação das subsubseções

\renewcommand{\cftsecnumwidth}{1em} % Reduz o espaço entre o número da seção e o título
\renewcommand{\cftsubsecnumwidth}{2em} % Para subseções
\renewcommand{\cftsubsubsecnumwidth}{3em} % Para subsubseções
\newcommand{\cftparagraphnumwidth}{4em} % Para paragrafos
\newcommand{\cftsubparagraphnumwidth}{5em} % Para sub-paragrafo

% Ajustar o espaçamento entre os itens do TOC
\setlength{\cftbeforesecskip}{10pt} % Espaçamento antes das seções
\setlength{\cftbeforesubsecskip}{2pt} % Espaçamento antes das subseções
\setlength{\cftbeforesubsubsecskip}{2pt} % Espaçamento antes das subsubseções

% Definição de comandos para seções adicionais no TOC (subsub, paragrafo e subparagrafo caso existam)
\newcommand{\cftparagraphpagefont}{\color{corSubsDois}\normalsize\bfseries}
\newcommand{\cftsubparagraphpagefont}{\color{corSubsDois}\normalsize\bfseries}
\newcommand{\cftparagraphfont}{\color{corSubsDois}\normalsize\bfseries}
\newcommand{\cftsubparagraphfont}{\color{corSubsDois}\normalsize\bfseries}

% Aplicar a formatação do texto aos números das páginas no TOC (subsub, paragrafo e subparagrafo caso existam)
\renewcommand{\cftsecpagefont}{\color{corSubsDois}\large\bfseries}  % H1
\renewcommand{\cftsubsecpagefont}{\color{corSubsDois}\normalsize} % H2
\renewcommand{\cftsubsubsecpagefont}{\color{corSubsDois}\normalsize}  % H3
\renewcommand{\cftparagraphpagefont}{\color{corSubsDois}\normalsize} % H4
\renewcommand{\cftsubparagraphpagefont}{\color{corSubsDois}\normalsize} % H5

% Estilo No TOC (subsub, paragrafo e subparagrafo caso existam)
\renewcommand{\cftsecfont}{%\color{corFundoCapa}
\large\bfseries}  % Estilo do H1
\renewcommand{\cftsubsecfont}{\color{corSubsDois}\normalsize} % Estilo do H2
\renewcommand{\cftsubsubsecfont}{\color{corSubsDois}\normalsize}  % Estilo do H3
\renewcommand{\cftparagraphfont}{\color{corSubsDois}\normalsize} % Estilo do H4
\renewcommand{\cftsubparagraphfont}{\color{corSubsDois}\normalsize} % Estilo do H5


% Define formatacao personalizada para seções - aqui ja é do documento qmd e nao do TOC especificamente
\titleformat{\section}
  {\normalfont\Huge\bfseries\color{corSubsDois}} % Formatação do título da seção
  {\thesection} % Número da seção
  {0.5\baselineskip} % Espaço entre o número e o título
  {\textbf{\huge}} % Formatação do título da seção
  [\vspace{1.5ex}] % Espaço após o número e antes do título

% Define formatacao personalizada para subsecs
\titleformat{\subsection}
  {\normalfont\Large\bfseries\color{corSubsDois}} % Formatação do título da subseção
  {\textbf{\thesubsection}} % Número da subseção
  {1ex} % Espaço entre o número e o título
  {\textbf{\Large}} % Formatação do título da subseção
  [\vspace{1ex}] % Espaço após o número e antes do título


% Define formatacao personalizada para subsubsecs
\titleformat{\subsubsection}
  {\normalfont\large\color{corSubsDois}} % Formatação do título da subsubseção
  {\thesubsubsection} % Número da subsubseção
  {1ex} % Espaço entre o número e o título
  {\normalsize\bfseries} % Formatação do título da subsubseção
  [\vspace{0.5ex}] % Espaço após o número e antes do título

% Define formatacao personalizada para paragrafos
\titleformat{\paragraph}
  {\normalfont\normalsize\tbBold\color{corSubsDois}} % Formatação do título do parágrafo
  {\theparagraph} % Número do parágrafo
  {1ex} % Espaço entre o número e o título
  {} % Formatação do título do parágrafo
  [\vspace{0.5ex}] % Espaço após o número e antes do título

% Define formatacao personalizada para subparagrafos
\titleformat{\cftsubparagraph}
  {\normalfont\normalsize\tbBold\color{corSubsDois}} % Formatação do título do subparágrafo
  {\textbf{\normalsize\textcolor{corSubsDois}{\arabic{subparagraph}}}} % Número do subparágrafo
  {1ex} % Espaço entre o número e o título
  {} % Formatação do título do subparágrafo
  [\vspace{0.5ex}] % Espaço após o número e antes do título



% Configuração para numeração romana e arábica
%\newcommand{\setRomanPageNumbering}{
%  \pagenumbering{Roman}
%}

\newcommand{\setRomanPageNumbering}{
  \renewcommand{\thepage}{\Roman{page}}%
  \pagenumbering{roman}
}


\newcommand{\setArabicPageNumbering}{
  \clearpage
  \pagenumbering{arabic}
}

% remover a linha horizontal no cabeçalho do fancy que estava confligando comimagem de fundo
\renewcommand{\headrulewidth}{0pt}
%\setArabicPageNumbering



\hypersetup{
  colorlinks=true,
  linkcolor=corTitulo
}
}



